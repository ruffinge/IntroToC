\documentclass[IntroToC.tex]{subfiles}

\begin{document}
\chapter{Basics}

We will begin by examining the structure of a basic C program. As is the 
long-established convention\footnote{This convention can be traced to the first edition
of Kernighan and Ritchie's \emph{C Programming Language}. See the note on 
other resource in the introduction.}, we will use a ``Hello World'' program to
introduce these concepts.

Examine the source file \texttt{hello.c} reproduced in \autoref{lst:hello.c}.

\begin{code}[caption=Source of the Hello World program.,label=lst:hello.c]
#include <stdio.h>

int main(void) {
	printf("Hello, World!");
}
\end{code}

\section{Preprocessor Commands}
We will first examine the preprocessor command at line one. This line instructs
the preprocessor to insert the entire source of the file \texttt{stdio.h} in the
current position.

That file (\texttt{stdio.h}) is a standard library file. It contains definitions
of various functions for interacting with users and file systems. (The file name
itself is an abbreviation of ``\textbf{st}andar\textbf{d}
\textbf{i}nput/\textbf{o}utput''.)

This command will automaticaly be processed before the source is passed to the
main compiler in order to ensure that all necessary functions have been defined.

\section{Function Definitions}
Next, we will examine lines three through five.

\end{document}
